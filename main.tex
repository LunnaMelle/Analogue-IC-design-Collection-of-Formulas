\documentclass[12pt, a4paper]{article}%
\usepackage[top=4cm, bottom=4cm, left=3cm, right=3cm]{geometry}
\usepackage{graphicx}
\usepackage{fancyhdr}
\usepackage{lastpage}
\usepackage{listings}
\usepackage{xcolor}
\usepackage{tikz}
\usepackage{float}
\usepackage{datetime2}
\usepackage{amsmath}
\allowdisplaybreaks
\usepackage{circuitikz}
\usepackage{import}
\usepackage{hhline}
\usepackage{amssymb}
\usepackage{icomma}
\usepackage{mathtools}
\usepackage{siunitx}

\usepackage{listings}
\usepackage{xcolor} % För färger om du vill ha syntax highlighting

\usepackage{caption}
\usepackage{subcaption}
\captionsetup{labelfont=bf}



\lstdefinestyle{matlabStyle}{
    language=Matlab,
    backgroundcolor=\color{gray!10},
    basicstyle=\ttfamily\footnotesize,
    keywordstyle=\color{blue},
    commentstyle=\color{green!50!black},
    stringstyle=\color{red},
    numbers=left,
    numberstyle=\tiny\color{gray},
    stepnumber=1,
    numbersep=5pt,
    breaklines=true,
    showstringspaces=false,
    captionpos=b
}
\lstset{style=matlabStyle}



\newcommand{\degree}{\ensuremath{^\circ}}


% ieee referatsystem
% \usepackage[backend=biber, style=ieee]{biblatex}
% \addbibresource{referenser.bib}

\setlength{\parindent}{0pt}
\setlength{\parskip}{0.75em}

% Hyperref should be loaded last
\usepackage{bookmark}

\usepackage{hyperref}
\hypersetup{
    colorlinks=true,
    linkcolor=black,
    urlcolor=blue,
    citecolor=blue
}

% Header setup
\pagestyle{fancy}
\fancyhead[R]{Analogue IC-design ETIN25 \hspace{1em} \thepage/\pageref{LastPage}}
\fancyfoot{}

\setlength{\headheight}{14.5pt} % Fix fancyhdr warning
\setcounter{secnumdepth}{0} % Only number up to subsections

\begin{document}

\begin{titlepage}
    \centering
    
    \vspace{8cm}
    
    % Titel
{\Huge \selectfont\bfseries Collection of Formulas \\ and compendium \par}

    \vspace{1cm}

    {\Large Analogue IC-design ETIN25}
    
    \vfill

    \includegraphics[width=0.5\textwidth]{pictures/Sigill_Lunds_universitet_(vit).png}\par

    {\textit{ \textbf{OBS!! Work in progress \\ Made by student Melker Rose E23}}}
    
    {\textit{Department of Electrical and Information Technology \\ Faculty of Engineering LTH \\ Lund University, Sweden \\ \DTMtoday}}

    \vspace{1cm}

    
\end{titlepage}

\thispagestyle{fancy}
\setcounter{page}{2}

\newpage

\section{Preface}
This collection of formulas is an unofficial formula sheet created by me, a student (Melker Rose E23), for the course Analogue IC-design (ETIN25) at LTH. 

The primary purposes of creating this document were:
\begin{itemize}
    \item To gather all relevant formulas in one convenient location for study and reference
    \item To enhance learning through the process of organizing and typesetting the material
\end{itemize}

\textbf{Important notes:}
\begin{itemize}
    \item This is \emph{not} an official document from the course administration, department, or faculty
    \item Although care has been taken to ensure accuracy and formulas are primarily sourced from official materials, the coursebook (\textbf{Gray, Hurst, Lewis, Meyer:} \textit{Analysis and Design of Analog Integrated Circuits}, Fifth Edition. Wiley 2010.) and/or lecture slides, cross-referencing is recommended, as there may be errors or omissions
    \item This document is a work in progress and may be updated
\end{itemize}

I hope this collection of formulas proves useful. \\ Best of luck, 

\textbf{\textit{Melker Rose} \hfill \textit{\DTMtoday}}

\newpage

\tableofcontents

\newpage

\section{Formulas}

\subsection{General formulas}
\begin{flalign}
\parbox{4.5cm}{Transconductance} 
& \quad G_m = \frac{i_o}{v_i} \Bigg|_{v_o=0} &&\\[4pt]
%
\parbox{4.5cm}{Input resistance\\(with test source)} 
& \quad R_i = \frac{v_t}{i_t} \Bigg|_{v_o=0} &&\\[4pt]
%
\parbox{4.5cm}{Output resistance\\(with test source)} 
& \quad R_o = \frac{v_t}{i_t} \Bigg|_{v_i=0} &&\\[4pt]
%
\parbox{4.5cm}{Voltage gain} 
& \quad A_v = G_mR_o &&\\[4pt]
%
\parbox{4.5cm}{Offset Voltage of the\\Source-Coupled Pair} 
& \quad V_{OS} = \Delta V_t + \frac{(V_{GS} - V_t)}{2} \left( -\frac{\Delta R_L}{R_L} - \frac{\Delta (W/L)}{(W/L)} \right) &&
\end{flalign}

\subsection{\textit{npn} Bipolar Transistor Parameters}

\subsubsection{Large-Signal Forward-Active Operation}

\begin{flalign}
\parbox{4.5cm}{Collector current} 
& \quad I_C = I_S \exp\left(\frac{V_{be}}{V_T}\right) &&
\end{flalign}

\subsubsection{Small-Signal Forward-Active Operation}

\begin{flalign}
\parbox{4.5cm}{Transconductance} 
& \quad g_m = \frac{qI_C}{kT} = \frac{I_C}{V_T} &&\\[4pt]
%
\parbox{4.5cm}{Transconductance-to-current ratio} 
& \quad \frac{g_m}{I_C} = \frac{1}{V_T} &&\\[4pt]
%
\parbox{4.5cm}{Input resistance} 
& \quad r_\pi = \frac{\beta_0}{g_m} &&\\[4pt]
%
\parbox{4.5cm}{Output resistance} 
& \quad r_o = \frac{V_A}{I_C} = \frac{1}{\eta g_m} &&\\[4pt]
%
\parbox{4.5cm}{Collector-base resistance} 
& \quad r_\mu = \beta_0 r_o \text{ to } 5\beta_0 r_o &&\\[4pt]
%
\parbox{4.5cm}{Base-charging capacitance} 
& \quad C_b = \tau_F g_m &&\\[4pt]
%
\parbox{4.5cm}{Base-emitter capacitance} 
& \quad C_\pi = C_b + C_{je} &&\\[4pt]
%
\parbox{4.5cm}{Emitter-base junction depletion capacitance} 
& \quad C_{je} \simeq 2C_{je0} &&\\[4pt]
%
\parbox{4.5cm}{Collector-base\\junction capacitance} 
& \quad C_{\mu} = \frac{C_{\mu 0}}{\left(1 - \frac{V_{BC}}{\psi_{0c}}\right)^{n_c}} &&\\[4pt]
%
\parbox{4.5cm}{Collector-substrate\\junction capacitance} 
& \quad C_{cs} = \frac{C_{cs 0}}{\left(1 - \frac{V_{SC}}{\psi_{0s}}\right)^{n_s}} &&\\[4pt]
%
\parbox{4.5cm}{Transition\\frequency} 
& \quad f_T = \frac{1}{2\pi}\frac{g_m}{C_x + C_{\mu}} &&\\[4pt]
%
\parbox{4.5cm}{Effective\\transit time} 
& \quad \tau_T = \frac{1}{2\pi f_T} = \tau_F + \frac{C_{je}}{g_m} + \frac{C_{\mu}}{g_m} &&\\[4pt]
%
\parbox{4.5cm}{Maximum\\gain} 
& \quad g_m r_o = \frac{V_A}{V_T} = \frac{1}{\eta} &&
\end{flalign}

\subsection{NMOS Transistor Parameters}

\subsubsection{Large-Signal Operation}

\begin{flalign}
\parbox{4.5cm}{Process transconductance parameter} 
& \quad k' = \mu C_{ox} &&\\[4pt]
%
\parbox{4.5cm}{Drain current (active region)} 
& \quad I_D = \frac{k'}{2}\frac{W}{L}(V_{GS} - V_t)^2 &&\\[4pt]
%
\parbox{4.5cm}{Overdrive voltage (active region)} 
& \quad V_{OV} = V_{GS} - V_t = \sqrt{\frac{I_D}{\frac{k'}{2}}\frac{W}{L}}  &&\\[4pt]
%
\parbox{4.5cm}{Early voltage} 
& \quad V_A = \frac{I_D}{\delta I_D / \delta V_{DS}} = L_{eff} \left( \frac{\delta X_d}{\delta V_{DS}} \right)^{-1} &&\\[4pt]
%
\parbox{4.5cm}{Reciprocal of the Early voltage} 
& \quad \lambda = \frac{1}{V_A} &&\\[4pt]
%
\parbox{4.5cm}{Drain current (active region) better} 
& \quad I_D = \frac{k'}{2}\frac{W}{L}(V_{GS} - V_t)^2(1+\lambda V_{DS}) &&\\[4pt]
%
\parbox{4.5cm}{Drain current (triode region)} 
& \quad I_D = \frac{k'}{2}\frac{W}{L}\left[2(V_{GS}-V_t)V_{DS} - V_{DS}^2\right] &&\\[4pt]
%
\parbox{4.5cm}{Characteristic current} 
& \quad I_t = qXD_n n_{po} \exp\left(\frac{k_2}{V_T}\right) &&\\[4pt]
%
\parbox{4.5cm}{Drain current\\ (subthreshold region)} 
& \quad I_D = \frac{W}{L} I_t \exp\left(\frac{V_{GS} - V_t}{nV_T}\right) \left[1 - \exp\left(-\frac{V_{DS}}{V_T}\right)\right] &&\\[4pt]
%
\parbox{4.5cm}{Threshold voltage}
& \quad V_t = V_{t0} + \gamma\left[\sqrt{2\phi_f + V_{SB}} - \sqrt{2\phi_f}\right] &&\\[4pt]
%
\parbox{4.5cm}{Threshold voltage parameter}
& \quad \gamma = \frac{1}{C_{ox}}\sqrt{2 q \varepsilon N_A} &&\\[4pt]
%
\parbox{4.5cm}{Oxide capacitance}
& \quad C_{ox} = \frac{\varepsilon_{ox}}{t_{ox}} 
= 3.45\,\text{fF}/\mu\text{m}^2 \quad \text{for } t_{ox}=100\,\text{\AA} &&
\end{flalign}

\subsubsection{Small-Signal Operation (Active Region)}

\begin{flalign}
\parbox{4.5cm}{Top-gate\\transconductance} 
& \quad g_m = \frac{\delta I_D}{\delta V_{GS}} = k'\frac{W}{L}(V_{GS}-V_t) = \sqrt{2 I_D k'\frac{W}{L}} &&\\[4pt]
%
\parbox{4.5cm}{Transconductance-to-\\current ratio} 
& \quad \frac{g_m}{I_D} = \frac{2}{V_{GS}-V_t} &&\\[4pt]
%
\parbox{4.5cm}{Body-effect\\transconductance} 
& \quad g_{mb} = \frac{\gamma}{2\sqrt{2\phi_f + V_{SB}}}g_m = \chi g_m &&\\[4pt]
%
\parbox{4.5cm}{Channel-length\\modulation parameter} 
& \quad \lambda = \frac{1}{V_A} = \frac{1}{L_\text{eff}}\,\frac{dX_d}{dV_{DS}} &&\\[4pt]
%
\parbox{4.5cm}{Output\\resistance} 
& \quad r_o = \frac{1}{\lambda I_D} = \frac{L_{\text{eff}}}{I_D} \left( \frac{dX_d}{dV_{DS}} \right)^{-1} &&\\[4pt]
%
\parbox{4.5cm}{Effective channel\\length} 
& \quad L_{\text{eff}} = L_{\text{down}} - 2L_d - X_d &&\\[4pt]
%
\parbox{4.5cm}{Maximum\\gain} 
& \quad g_m r_o = \frac{1}{\lambda} \frac{2}{V_{GS} - V_i} = \frac{2V_A}{V_{GS} - V_i} &&\\[4pt]
%
\parbox{4.5cm}{Source-body\\depletion capacitance} 
& \quad C_{sb} = \frac{C_{s0}}{\left( 1 + \frac{V_{SB}}{\psi_0} \right)^{0.5}} &&\\[4pt]
%
\parbox{4.5cm}{Drain-body\\depletion capacitance} 
& \quad C_{db} = \frac{C_{db0}}{\left( 1 + \frac{V_{DB}}{\psi_0} \right)^{0.5}} &&\\[4pt]
%
\parbox{4.5cm}{Gate-source\\capacitance} 
& \quad C_{gs} = \frac{2}{3} WL C_{ox} + (C_{gd}) &&\\[4pt]
%
\parbox{4.5cm}{Gate-drain\\capacitance} 
& \quad C_{gd} = W C_{ol} &&\\[4pt]
%
\parbox{4.5cm}{Transition\\frequency} 
& \quad f_T = \frac{g_m}{2\pi (C_{gs} + C_{sd} + C_{gb})} &&\\[4pt]
%
\parbox{4.5cm}{Miller\\capacitance} 
& \quad C_M = (1-A_V)C_{gd} &&
%
\end{flalign}

\subsection{Frequency}
\begin{flalign}
\parbox{4.5cm}{-$3$dB frequency} 
& \quad f_{-3\text{dB}} =  &&\\[4pt]
%
\parbox{4.5cm}{A\% to B\% rise time} 
& \quad t_{r} = \frac{\ln \left( \frac{B\%}{A\%} \right)}{2\pi} \frac{1}{f_{-3\text{dB}}} &&\\[4pt]
\end{flalign}


\begin{flalign}
\parbox{4.5cm}{Example\\example} 
& \quad A = \frac{B}{C} &&\\[4pt]
\end{flalign}

\begin{figure}[H]
    \centering
    
\begin{circuitikz}[american]
    % Global styles for consistent sizing
    \ctikzset{bipoles/length=1.2cm}

    % Transistor (N-channel MOSFET)
    % Placed at origin for relative positioning
    \draw (0,0) node[nmos] (Q) {};

    % Input Section (Gate)
    % Wire goes left then down through voltage source to ground
    \draw (Q.G) -- (-2.5, 0 |- Q.G) coordinate (in_top)
          to[european voltage source, l_=$V_i$] (-2.5, -2.5) node[ground] {};
    
    % Input Current Arrow (Ii)
    \draw[-Latex] (-2.0, 0.3) -- (-1.0, 0.3) node[midway, above] {$I_i$};
    
    % Input Polarity Signs (Manual placement to match image style)
    \node[left] at (-2.7, -0.5) {$+$};
    \node[left] at (-2.7, -2.0) {$-$};

    % Drain Section
    % Wire goes up through resistor to VDD
    \draw (Q.D) node[circ] {} -- (0, 2.0) to[R, l_=$R_D$] (0, 4.5) coordinate (vdd);
    % VDD Rail
    \draw [thick] (-0.8, 4.5) -- (0.8, 4.5) node[midway, above] {$V_{DD}$};
    
    % Drain Current Arrow (Id)
    \draw[-Latex] (0.6, 1.8) -- (0.6, 0.8) node[midway, right] {$I_d$};

    % Source Section
    % Wire goes down to ground
    \draw (Q.S) -- (0, -2.5) coordinate (s_node);
    \draw (s_node) node[circ] {} node[ground] {};

    % Output Section
    % Top Terminal connected to Drain
    \draw (Q.D) -- (3.5, 0 |- Q.D) node[ocirc] (out_top) {};
    % Bottom Terminal connected to Source/Ground
    \draw (s_node) -- (3.5, -2.5) node[ocirc] (out_bot) {};

    % Output Current Arrow (Io) - Pointing Left
    \draw[-Latex] (3.0, 1) -- (2.0, 1) node[midway, above] {$I_o$};

    % Output Voltage Label (Vo)
    \node at (3.5, -1.0) {$V_o$};
    \node[below] at (out_top) {$+$};
    \node[above] at (out_bot) {$-$};

\end{circuitikz}
 % Inserts the LaTeX file
    \caption{Common-source amplifier stage}
    \label{fig:CS-stage}
\end{figure}

\section{Amplifying stages}

\subsection{Single transistor amplifying stages}

\subsubsection{Common source (MOSFET)}

\begin{figure}[H]
  \centering
  % First Subfigure
  \begin{subfigure}{0.45\textwidth}
    \centering
    \includegraphics[width=\linewidth]{pictures/Resistively loaded common source.png}
    \caption{Resistively loaded, common-source amplifier}
    \label{fig:Resistively loaded, common-source amplifier}
  \end{subfigure}
  \hfill
  % Second Subfigure
  \begin{subfigure}{0.5\textwidth}
    \centering
    \includegraphics[width=\linewidth]{pictures/small signal eq common source.png}
    \caption{Low frequency small-signal equivalent circuit for the
common-source amplifier}
    \label{fig:small signal eq common source}
  \end{subfigure}
  \caption{Common-source amplifier}
  \label{fig:Common-source stage (MOSFET)}
\end{figure}

The transconductance of the CS amplifier can be expressed as:
\begin{equation}
G_m = \frac{i_o}{v_i} \bigg|_{v_o = 0} = g_m
\end{equation}
The (small signal) input resistance \( R_i \) is
\begin{equation}
R_i = \frac{v_i}{i_i} \to \infty
\end{equation}
The output resistance is
\begin{equation}
R_o = \frac{v_o}{i_o} \bigg|_{v_i = 0} = R_D \parallel r_o
\end{equation}
The open-circuit, or unloaded, voltage gain is
\begin{equation}
a_v = \frac{v_o}{v_i} \bigg|_{i_o = 0} = -g_m(r_o \parallel R_D)
\end{equation}

If the drain load resistor \( R_D \) is replaced by a current source, \( R_D \to \infty \) and \( a_v \) becomes

\begin{equation}
\lim_{R_D \to \infty} a_v = -g_mr_o
\end{equation}

\subsubsection{Common gate (MOSFET)}
\begin{figure}[H]
  \centering
  % First Subfigure
  \begin{subfigure}{0.45\textwidth}
    \centering
    \includegraphics[width=\linewidth]{pictures/Common gate.png}
    \caption{Common-gate configuration}
    \label{fig:Common gate}
  \end{subfigure}
  \hfill
  % Second Subfigure
  \begin{subfigure}{0.5\textwidth}
    \centering
    \begin{circuitikz}
\tikzstyle{every node}=[font=\normalsize]

\draw 
         (0,2) to[short, o-, l=$v_i$] (1,2) to [R, l=$r_o$] (3,2) to [short] (4,2) to [R, l=$R_D$] (4,0)

         (5,2) to [short, o-, l_=$v_o$, i=$i_o$] (4,2)

        (1,2) to (1,1) to [american current source,l_={ \normalsize $(g_m + g_{mb})v_i$}] (3,1) to (3,2)
        
        
        (4,0) node[ground]{};
\end{circuitikz}
    \caption{Low frequency small-signal equivalent circuit for the
common-gate stage}
    \label{fig:Small signal common gate}
  \end{subfigure}
  \caption{Common-gate stage}
  \label{fig:Common-gate stage (MOSFET)}
\end{figure}
The transconductance of the CG-stage can be expressed as:
\begin{equation}
G_m = \frac{i_o}{v_i} = g_m + g_{mb} + \frac{1}{r_o} \approx g_m + g_{mb}
\end{equation}
The output resistance
\begin{equation}
R_o = \frac{v_o}{i_o} = r_o \parallel R_D
\end{equation}
The open-circuit voltage
\begin{equation}
A_v = \frac{v_o}{v_i} = +\left(g_m + g_{mb}\right)\left(r_o \parallel R_D\right) \xrightarrow[r_o \gg R_D]{} \left(g_m + g_{mb}\right)R_D
\end{equation}
The input resistance can be expressed as
\begin{equation}
R_i = \frac{v_i}{i_i} = \frac{R_D + r_o}{1 + \left( g_m + g_{mb} \right) r_o}
\end{equation}
which if \( r_o >> R_D\) and \(g_mr_o >> 1\) then becomes
\begin{equation}
R_i \approx \frac{1}{g_m + g_{mb}} \approx \frac{1}{g_m}
\end{equation}

\subsubsection{Common Drain (MOSFET)}
\begin{figure}[H]
  \centering
  % First Subfigure
  \begin{subfigure}{0.39\textwidth}
    \centering
    \includegraphics[width=\linewidth]{pictures/Common-drain configuration.png}
    \caption{Common-drain configuration}
    \label{fig:Common drain}
  \end{subfigure}
  \hfill
  % Second Subfigure
  \begin{subfigure}{0.6\textwidth}
    \centering
    \includegraphics[width=\linewidth]{pictures/Small signal common drain config.png}
    \caption{Small-signal
equivalent circuit of the common-drain configuration}
    \label{fig:Small signal common drain config}
  \end{subfigure}
  \caption{Common-drain stage also called source follower}
  \label{fig:Common-drain stage (MOSFET)}
\end{figure}

The transconductance of the CD-stage can be expressed as:
\begin{equation}
    G_m = g_m
\end{equation}
and the output resistance
\begin{equation}
R_o = \frac{v_o}{i_o} = \frac{1}{g_m + g_{mb} + \frac{1}{r_o} + \frac{1}{R_L}}
\end{equation}
Which then gives
\begin{equation}
A_v = \frac{g_m}{g_m + g_{mb} + \frac{1}{R_L} + \frac{1}{r_o}} = \frac{g_m R_L}{1 + \left( g_m + g_{mb} \right) R_L + \frac{R_L}{r_o}} < 1
\end{equation}
The input resistance (small-signal) is \(R_i = \infty\)

\subsubsection{Common source with source degeneration (MOSFET)}
\begin{figure}[H]
  \centering
  % First Subfigure
  \begin{subfigure}{0.39\textwidth}
    \centering
    \includegraphics[width=\linewidth]{pictures/Common-source amp with source degeneration.png}
    \caption{Common-source amplifier with source degeneration}
    \label{fig:Common-source amp with source degeneration}
  \end{subfigure}
  \hfill
  % Second Subfigure
  \begin{subfigure}{0.6\textwidth}
    \centering
    \includegraphics[width=\linewidth]{pictures/Small signal common-source amp with source degeneration.png}
    \caption{Small-signal equivalent of the source-degenerated, common-source amplifier}
    \label{fig:Small signal common-source amp with source degeneration}
  \end{subfigure}
  \caption{Source-degenerated, common-source amplifier}
  \label{fig:Common-source source degenerated (MOSFET)}
\end{figure}

The transconductance of the source-degenerated, common-source amplifier can be expressed as:
\begin{equation}
G_m = \frac{i_o}{v_i} = \frac{g_m}{1 + (g_m + g_{mb}) \, R_S + \frac{R_S}{r_o}}
\end{equation}
which if \( r_o \gg R_S \),
\begin{equation}
G_m \simeq \frac{g_m}{1 + (g_m + g_{mb}) \, R_S}
\end{equation}
The output resistance becomes
\begin{equation}
R_o = \frac{v_t}{i_t} = R_S + r_o \left[ 1 + (g_m + g_{mb}) R_S \right]
\end{equation}

\subsection{Multiple transistor amplifying stages}

\subsubsection{Cascode amplifier (CS-CG)}
\begin{figure}[H]
  \centering
  % First Subfigure
  \begin{subfigure}{0.49\textwidth}
    \centering
    \includegraphics[width=\linewidth]{pictures/cascode amplifier using mosfet.png}
    \caption{Cascode amplifier using MOSFET (CS-CG)}
    \label{fig:Common-source amp with source degeneration}
  \end{subfigure}
  \hfill
  % Second Subfigure
  \begin{subfigure}{0.5\textwidth}
    \centering
    \includegraphics[width=\linewidth]{pictures/Small signal ew of cascode amp using mosfet.png}
    \caption{Small-signal equivalent circuit for the MOS-transistor cascode connection}
    \label{fig:Small signal common-source amp with source degeneration}
  \end{subfigure}
  \caption{CS-stage followed by CG-stage cascode amplifier}
  \label{fig:Common-source source degenerated (MOSFET)}
\end{figure}


\section{Methods}

\subsection{Dominant-Pole Approximation}

For any electronic circuit, a transfer function can be derived
\begin{equation}
A(s) = \frac{a_0 + a_1 s + a_2 s^2 + \cdots + a_m s^m}{1 + b_1 s + b_2 s^2 + \cdots + b_n s^n}
\end{equation}
this can be factored into
\begin{equation}
    A(s) = K \frac{\left( 1 - \frac{s}{n_1} \right) \left( 1 - \frac{s}{n_2} \right) \cdots \left( 1 - \frac{s}{n_m} \right)}{\left( 1 - \frac{s}{p_1} \right) \left( 1 - \frac{s}{p_2} \right) \cdots \left( 1 - \frac{s}{p_n} \right)} 
\end{equation}
where \( K \) is a constant, \(n_i\) are the zeros, and \( p_i\) are the poles of the transfer function and
\begin{equation}
b_1 = \sum_{i=1}^n \left( -\frac{1}{p_i} \right)
\end{equation}
When one pole is dominant, i.e
\begin{equation}
|p_1| \ll |p_2|, |p_3|, \ldots \quad \text{so that} \quad \left| \frac{1}{p_1} \right| \gg \left| \sum_{i=2}^n \left( -\frac{1}{p_i} \right) \right|
\end{equation}
it can be shown that
\begin{equation}
b_1 \simeq \left| \frac{1}{p_1} \right|
\end{equation}

\subsubsection{Open-circuit (zero-value) time-constant}
An approximation of \(b_1\) is 
\begin{align}
b_{1}=\sum_{i=1}^{n}\tau_{i}=\sum_{i=1}^{n}R_{i0}C_{i}
\end{align}
where $R_{i0}$ is the resistance seen by $C_i$ (i.e., with $C_i$ working as a voltage or current source) when all other capacitors are removed (open circuits, zeroed), and all independent voltage and current sources are zeroed as well; i.e., independent voltage sources become short circuits, and independent current sources become open circuits.

For this the \(-3\text{dB}\) frequency can be estimated as
\begin{equation}
    f_{-3\text{dB}} = \frac{\omega_{-3\text{dB}}}{2\pi} \approx \frac{1}{2\pi b_1} = \frac{1}{2\pi \sum_{\forall i} \tau_i}
\end{equation}

\subsubsection{Open-circuit time-constant example}
Calculate the \(-3\text{dB}\) angular frequency of the following circuit (ignore $C_{gb}$)
\begin{figure}[H]
    \centering
    \includegraphics[width=0.7\linewidth]{pictures/Two-stage common-source cascade amplifier.png}
    \caption{Two-stage common-source cascade amplifier}
    \label{fig:Two-stage common-source cascade amplifier}
\end{figure}
\begin{figure}[H]
    \centering
    \includegraphics[width=1\linewidth]{pictures/Small-signal equivalent circuit of two-stage common-source cascade amplifier.png}
    \caption{Small-signal equivalent circuit of fig. \ref{fig:Two-stage common-source cascade amplifier}}
    \label{fig:Small-signal equivalent circuit of two-stage common-source cascade amplifier}
\end{figure}

By using the open-circuit time-constant it can be found that
\begin{align}
    \tau_{gd1} &= C_{gd1} R_{gd01} = C_{gd1} (R_S + R_{L1} + g_{m1} R_{L1} R_S) \\
    \tau_{gd2} &=C_{gd2} R_{gd02} = C_{gd2} (R_{L1} + R_{L2} + g_{m2} R_{L2} R_{L1}) \\
    \tau_{gs1} &= C_{gs1} R_{gs01} = C_{gs1} R_S \\
    \tau_{gs2} &= C_{gs2} R_{gs02} = C_{gs1} R_{L1} \\
    \tau_{db1} &= C_{db1} R_{db01} = C_{db1} R_{L1} \\
    \tau_{db2} &= C_{db2} R_{db02} = C_{db2} R_{L2}
\end{align}
for which
\begin{equation}
    b_1 = \sum_{\forall i} \tau_i = \sum_{i=1}^6 \tau_i
\end{equation}
and thus 
\begin{equation}
    \omega_{-3\text{dB}} \approx \frac{1}{b_1} = \frac{1}{\sum_{\forall i} \tau_i}
\end{equation}



\end{document}