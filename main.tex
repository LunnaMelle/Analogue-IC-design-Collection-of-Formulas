\documentclass[12pt, a4paper]{article}%
\usepackage[top=4cm, bottom=4cm, left=3cm, right=3cm]{geometry}
\usepackage{graphicx}
\usepackage{fancyhdr}
\usepackage{lastpage}
\usepackage{listings}
\usepackage{xcolor}
\usepackage{tikz}
\usepackage{float}
\usepackage{datetime2}
\usepackage{amsmath}
\allowdisplaybreaks
\usepackage{circuitikz}
\usepackage{import}
\usepackage{hhline}
\usepackage{amssymb}
\usepackage{icomma}
\usepackage{mathtools}
\usepackage{siunitx}

\usepackage{listings}
\usepackage{xcolor} % För färger om du vill ha syntax highlighting

\usepackage{caption}
\captionsetup{labelfont=bf}



\lstdefinestyle{matlabStyle}{
    language=Matlab,
    backgroundcolor=\color{gray!10},
    basicstyle=\ttfamily\footnotesize,
    keywordstyle=\color{blue},
    commentstyle=\color{green!50!black},
    stringstyle=\color{red},
    numbers=left,
    numberstyle=\tiny\color{gray},
    stepnumber=1,
    numbersep=5pt,
    breaklines=true,
    showstringspaces=false,
    captionpos=b
}
\lstset{style=matlabStyle}



\newcommand{\degree}{\ensuremath{^\circ}}


% ieee referatsystem
% \usepackage[backend=biber, style=ieee]{biblatex}
% \addbibresource{referenser.bib}

\setlength{\parindent}{0pt}
\setlength{\parskip}{0.75em}

% Hyperref should be loaded last
\usepackage{hyperref}
\hypersetup{
    colorlinks=true,
    linkcolor=black,
    urlcolor=blue,
    citecolor=blue
}

% Header setup
\pagestyle{fancy}
\fancyhead[R]{Analogue IC-design ETIN25 \hspace{1em} \thepage/\pageref{LastPage}}
\fancyfoot{}

\setlength{\headheight}{14.5pt} % Fix fancyhdr warning
\setcounter{secnumdepth}{0} % Only number up to subsections

\begin{document}

\begin{titlepage}
    \centering
    
    \vspace{8cm}
    
    % Titel
{\fontsize{100pt}{120pt}\selectfont\bfseries Collection of Formulas\par}

    \vspace{1cm}

    {\Large Analogue IC-design ETIN25}
    
    \vfill

    \includegraphics[width=0.5\textwidth]{Sigill_Lunds_universitet_(vit).png}\par

    {\textit{ \textbf{OBS!! Work in progress \\ Made by student Melker Rose E23}}}
    
    {\textit{Department of Electrical and Information Technology \\ Faculty of Engineering LTH \\ Lund University, Sweden \\ \DTMtoday}}

    \vspace{1cm}

    
\end{titlepage}

\thispagestyle{fancy}
\setcounter{page}{2}

\newpage

\section{Preface}
This collection of formulas is an unofficial formula sheet created by me, a student (Melker Rose E23), for the course Analogue IC-design (ETIN25) at LTH. 

The primary purposes of creating this document were:
\begin{itemize}
    \item To gather all relevant formulas in one convenient location for study and reference
    \item To enhance learning through the process of organizing and typesetting the material
\end{itemize}

\textbf{Important notes:}
\begin{itemize}
    \item This is \emph{not} an official document from the course administration, department, or faculty
    \item Although care has been taken to ensure accuracy and formulas are primarily sourced from official materials, the coursebook (\textbf{Gray, Hurst, Lewis, Meyer:} \textit{Analysis and Design of Analog Integrated Circuits}, Fifth Edition. Wiley 2010.) and/or lecture slides, cross-referencing is recommended, as there may be errors or omissions
    \item This document is a work in progress and may be updated
\end{itemize}

I hope this collection of formulas proves useful. \\ Best of luck, 

\textbf{\textit{Melker Rose} \hfill \textit{\DTMtoday}}

\newpage

\tableofcontents

\newpage

\section{Active-Device Parameters}

\subsection{\textit{npn} Bipolar Transistor Parameters}

\subsubsection{Large-Signal Forward-Active Operation}

\begin{flalign}
\parbox{4.5cm}{Collector current} 
& \quad I_C = I_S \exp\left(\frac{V_{be}}{V_T}\right) &&
\end{flalign}

\subsubsection{Small-Signal Forward-Active Operation}

\begin{flalign}
\parbox{4.5cm}{Transconductance} 
& \quad g_m = \frac{qI_C}{kT} = \frac{I_C}{V_T} &&\\[4pt]
%
\parbox{4.5cm}{Transconductance-to-current ratio} 
& \quad \frac{g_m}{I_C} = \frac{1}{V_T} &&\\[4pt]
%
\parbox{4.5cm}{Input resistance} 
& \quad r_\pi = \frac{\beta_0}{g_m} &&\\[4pt]
%
\parbox{4.5cm}{Output resistance} 
& \quad r_o = \frac{V_A}{I_C} = \frac{1}{\eta g_m} &&\\[4pt]
%
\parbox{4.5cm}{Collector-base resistance} 
& \quad r_\mu = \beta_0 r_o \text{ to } 5\beta_0 r_o &&\\[4pt]
%
\parbox{4.5cm}{Base-charging capacitance} 
& \quad C_b = \tau_F g_m &&\\[4pt]
%
\parbox{4.5cm}{Base-emitter capacitance} 
& \quad C_\pi = C_b + C_{je} &&\\[4pt]
%
\parbox{4.5cm}{Emitter-base junction depletion capacitance} 
& \quad C_{je} \simeq 2C_{je0} &&\\[4pt]
%
\parbox{4.5cm}{Collector-base\\junction capacitance} 
& \quad C_{\mu} = \frac{C_{\mu 0}}{\left(1 - \frac{V_{BC}}{\psi_{0c}}\right)^{n_c}} &&\\[4pt]
%
\parbox{4.5cm}{Collector-substrate\\junction capacitance} 
& \quad C_{cs} = \frac{C_{cs 0}}{\left(1 - \frac{V_{SC}}{\psi_{0s}}\right)^{n_s}} &&\\[4pt]
%
\parbox{4.5cm}{Transition\\frequency} 
& \quad f_T = \frac{1}{2\pi}\frac{g_m}{C_x + C_{\mu}} &&\\[4pt]
%
\parbox{4.5cm}{Effective\\transit time} 
& \quad \tau_T = \frac{1}{2\pi f_T} = \tau_F + \frac{C_{je}}{g_m} + \frac{C_{\mu}}{g_m} &&\\[4pt]
%
\parbox{4.5cm}{Maximum\\gain} 
& \quad g_m r_o = \frac{V_A}{V_T} = \frac{1}{\eta} &&
\end{flalign}

\subsection{NMOS Transistor Parameters}

\subsubsection{Large-Signal Operation}

\begin{flalign}
\parbox{4.5cm}{Process transconductance parameter} 
& \quad k' = \mu C_{ox} &&\\[4pt]
%
\parbox{4.5cm}{Drain current (active region)} 
& \quad I_D = \frac{k'}{2}\frac{W}{L}(V_{GS} - V_t)^2 &&\\[4pt]
%
\parbox{4.5cm}{Early voltage} 
& \quad V_A = \frac{I_D}{\delta I_D / \delta V_{DS}} = L_{eff} \left( \frac{\delta X_d}{\delta V_{DS}} \right)^{-1} &&\\[4pt]
%
\parbox{4.5cm}{Reciprocal of the Early voltage} 
& \quad \lambda = \frac{1}{V_A} &&\\[4pt]
%
\parbox{4.5cm}{Drain current (active region) better} 
& \quad I_D = \frac{k'}{2}\frac{W}{L}(V_{GS} - V_t)^2(1+\lambda V_{DS}) &&\\[4pt]
%
\parbox{4.5cm}{Drain current (triode region)} 
& \quad I_D = \frac{k'}{2}\frac{W}{L}\left[2(V_{GS}-V_t)V_{DS} - V_{DS}^2\right] &&\\[4pt]
%
\parbox{4.5cm}{Characteristic current} 
& \quad I_t = qXD_n n_{po} \exp\left(\frac{k_2}{V_T}\right) &&\\[4pt]
%
\parbox{4.5cm}{Drain current\\ (subthreshold region)} 
& \quad I_D = \frac{W}{L} I_t \exp\left(\frac{V_{GS} - V_t}{nV_T}\right) \left[1 - \exp\left(-\frac{V_{DS}}{V_T}\right)\right] &&\\[4pt]
%
\parbox{4.5cm}{Threshold voltage}
& \quad V_t = V_{t0} + \gamma\left[\sqrt{2\phi_f + V_{SB}} - \sqrt{2\phi_f}\right] &&\\[4pt]
%
\parbox{4.5cm}{Threshold voltage parameter}
& \quad \gamma = \frac{1}{C_{ox}}\sqrt{2 q \varepsilon N_A} &&\\[4pt]
%
\parbox{4.5cm}{Oxide capacitance}
& \quad C_{ox} = \frac{\varepsilon_{ox}}{t_{ox}} 
= 3.45\,\text{fF}/\mu\text{m}^2 \quad \text{for } t_{ox}=100\,\text{\AA} &&
\end{flalign}

\subsubsection{Small-Signal Operation (Active Region)}

\begin{flalign}
\parbox{4.5cm}{Top-gate\\transconductance} 
& \quad g_m = \frac{\delta I_D}{\delta V_{GS}} = k'\frac{W}{L}(V_{GS}-V_t) = \sqrt{2 I_D k'\frac{W}{L}} &&\\[4pt]
%
\parbox{4.5cm}{Transconductance-to-\\current ratio} 
& \quad \frac{g_m}{I_D} = \frac{2}{V_{GS}-V_t} &&\\[4pt]
%
\parbox{4.5cm}{Body-effect\\transconductance} 
& \quad g_{mb} = \frac{\gamma}{2\sqrt{2\phi_f + V_{SB}}}g_m = \chi g_m &&\\[4pt]
%
\parbox{4.5cm}{Channel-length\\modulation parameter} 
& \quad \lambda = \frac{1}{V_A} = \frac{1}{L_\text{eff}}\,\frac{dX_d}{dV_{DS}} &&\\[4pt]
%
\parbox{4.5cm}{Output\\resistance} 
& \quad r_o = \frac{1}{\lambda I_D} = \frac{L_{\text{eff}}}{I_D} \left( \frac{dX_d}{dV_{DS}} \right)^{-1} &&\\[4pt]
%
\parbox{4.5cm}{Effective channel\\length} 
& \quad L_{\text{eff}} = L_{\text{down}} - 2L_d - X_d &&\\[4pt]
%
\parbox{4.5cm}{Maximum\\gain} 
& \quad g_m r_o = \frac{1}{\lambda} \frac{2}{V_{GS} - V_i} = \frac{2V_A}{V_{GS} - V_i} &&\\[4pt]
%
\parbox{4.5cm}{Source-body\\depletion capacitance} 
& \quad C_{sb} = \frac{C_{s0}}{\left( 1 + \frac{V_{SB}}{\psi_0} \right)^{0.5}} &&\\[4pt]
%
\parbox{4.5cm}{Drain-body\\depletion capacitance} 
& \quad C_{db} = \frac{C_{db0}}{\left( 1 + \frac{V_{DB}}{\psi_0} \right)^{0.5}} &&\\[4pt]
%
\parbox{4.5cm}{Gate-source\\capacitance} 
& \quad C_{gs} = \frac{2}{3} WL C_{ox} &&\\[4pt]
%
\parbox{4.5cm}{Transition\\frequency} 
& \quad f_T = \frac{g_m}{2\pi (C_{gs} + C_{sd} + C_{gb})} &&
%
\end{flalign}

\begin{flalign}
\parbox{4.5cm}{Transconductance} 
& \quad G_m = \frac{i_o}{v_i} \Bigg|_{v_o=0} &&\\[4pt]
\parbox{4.5cm}{Input resistance\\(with test source)} 
& \quad R_i = \frac{v_t}{i_t} \Bigg|_{v_o=0} &&\\[4pt]
\parbox{4.5cm}{Output resistance\\(with test source)} 
& \quad R_o = \frac{v_t}{i_t} \Bigg|_{v_i=0} &&\\[4pt]
\parbox{4.5cm}{Maximum gain} 
& \quad |A_v|_{max} = G_mR_o
\end{flalign}

\begin{flalign}
\parbox{4.5cm}{Example\\example} 
& \quad A = \frac{B}{C} &&\\[4pt]
\end{flalign}

\begin{flalign}
\parbox{4.5cm}{Example\\example} 
& \quad A = \frac{B}{C} &&\\[4pt]
\end{flalign}
\begin{flalign}
\parbox{4.5cm}{Example\\example} 
& \quad A = \frac{B}{C} &&\\[4pt]
\end{flalign}
\begin{flalign}
\parbox{4.5cm}{Example\\example} 
& \quad A = \frac{B}{C} &&\\[4pt]
\end{flalign}
\begin{flalign}
\parbox{4.5cm}{Example\\example} 
& \quad A = \frac{B}{C} &&\\[4pt]
\end{flalign}


\end{document}